\documentclass[11pt]{article}
\usepackage{../EllioStyle}

\title{Homework 1}
\author{Elliott Pryor}
\date{28 Aug 2020}


\begin{document}
\maketitle

\problem{1.22}
Prove that the sum of the deviations of a set of measurments about their mean is equal to zero:
$$\sum_{i=1}^n (y_i - \bar{y}) = 0$$
\hrule

\begin{proof}
We show that the sum of the deviations of a set about the mean is equal to zero. We first give the definition of $\bar{y} = 1/n \sum_{i=1}^n y_i$. For conciseness, we also let $S = \sum_{i=1}^n y_i$

\begin{align*}

\sum_{i=1}^n (y_i - \bar{y}) &= 0\\
\sum_{i=1}^n y_i - \sum_{i=1}^n \bar{y} &= 0\\
S - \sum_{i=1}^n (1/n \sum_{i=1}^n y_i) &= 0\\
S - 1/n \sum_{i=1}^n S) &= 0\\
S - 1/n * n * S) &= 0\\
S - S &= 0\\
0 &= 0
\end{align*}

\end{proof}




\problem{2.23}
If A and B are events and $B \subset A$, why is it ``obvious" that $P(B) \leq P(A)$.
\hrule

Qualitatively if $B \subset A$ then everything in $B$ is also in $A$. So the probability that $A$ happens must be at least as big as $B$. 

Mathematically, we can also show this.
\begin{proof}
 We let $C = A \setminus B$. Then $A = B \cup C$, and $B \cap C = \emptyset$. So we can use the third axiom of probability to state that $P(A) = P(B) + P(C)$. Well $P(A) = P(B) + P(C) \geq P(B)$ as required.
\end{proof}
 
 
 
 
\problem{2.33}
The Bureau of the Census reports that the median family income for all families in the United
States during the year 2003 was \$43,318. That is, half of all American families had incomes
exceeding this amount, and half had incomes equal to or below this amount. Suppose that four
families are surveyed and that each one reveals whether its income exceeded \$43,318 in 2003
\begin{enumerate}[\textbf{a}]
 	\item List the points in the sample space
 	\item Identify the simple events in each of the following spaces.
 	
 	
		\quad $A$: \quad At least two had incomes exceeding \$43,318 	
 	
 		\quad $B$: \quad Exactly two had incomes exceeding \$43,318
 	
 		\quad $C$: \quad Exactly two had incomes less than or equal to \$43,318
 	
 	\item Make use of the given interpretation for the median to assign probabilities to simple events and find $P(A), P(B), P(C)$  
\end{enumerate}
\hrule

\begin{enumerate}[a]

	\item We say that $E$ represents if a family exceeded the median and $N$ if they did not exceed the median income. 
	
	$\mathcal{S} = \{EEEE, EEEN, EENE, EENN, ENEE, ENEN, ENNE, ENNN, NEEE, NEEN, NENE, NENN, NNEE, NNEN, NNNE, NNNN\}$
	
	
	\item
	
	\quad $A = \{EEEE, EEEN, EENE, EENN, ENEE, ENEN, ENNE, NEEE, NEEN, NENE, NNEE\}$
	
	\quad $B = \{EENN, ENEN, ENNE, NEEN, NENE, NNEE\}$
	
	\quad $C = \{EEEN, EENE, ENEE, NEEE\}$
	
	
	\item By definition of the median value, exactly half of the individuals will have incomes exceeding the median amount. So the probability that a family has $E$ is $0.5$. We assume that all families are equally likely to be above the median value. So all simple events have a probability of $0.5 * 0.5 *0.5 * 0.5 = 1/16$.
	
	$P(A) = 1/16 * |A| = 1/16 * 11 = 0.6875$ 
	
	$P(B) = 1/16 * |B| = 1/16 * 6 = 0.375$ 
	
	$P(C) = 1/16 * |C| = 1/16 * 4 = 0.25$ 
	
	\problem{6} Do R tutorial.
	\hrule
	No questions.

\end{enumerate}

\end{document}


