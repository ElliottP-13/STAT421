\documentclass[11pt]{article}
\usepackage{./EllioStyle}

\title{Homework 0}
\author{Elliott Pryor}
\date{21 Aug 2020}


\begin{document}
\maketitle

\problem{2}

Chapter 1 of your text presents many definitions of statistics. How would you define statistics? Explain why this is your definition. Based on your definition, what role do you think mathematics plays in statistics?
\hrule

\begin{enumerate}[a)]
	\item Statistics is the quantitative study of data and distributions. It is about learning patterns that data follows and properties of the patterns.
	\item Mathematics is essential for providing a quantitative analysis. It provides tools for studying different properties of the distributions. 
\end{enumerate}

\problem{3}

\begin{enumerate}[a)]
	\item What issues did you encounter while working through the example code? If you did not encounter any issues please say so.
	\item Explain in your own words, how much experience you have had using R. Do you have concerns about executing code like that in these examples and the Real or Fake activity (aside from writing your own loops and apply functions) as we progress through class?
\end{enumerate}
\hrule

\begin{enumerate}[a)]
	\item I had no issues. It was strange to give plots that fail in the example, but it shows what errors are like I suppose.
	
	\item I have had no experience using R (besides the example). I am a computer science major, so I have no problems executing+writing my own code. 

\end{enumerate}


\problem{4}
Please share any questions or concerns you have with me about how class went for you the first week. You may share concerns about the workload, the use of class time, questions about the organization of content on the course website, or anything else that comes to mind that you feel might help you succeed in this course, and explain why. If you have no concerns please share something that is working well for you in this class so far, and explain why
\hrule

So far the class is pretty good. In all honesty,  probably not my favourite. I mostly don't like the flipped classroom. I personally would learn well if the synchronous activities were like the videos. So far, I am yet to get much educational value from the in class activities. It just doesn't seem like I learn or get to apply much. To be fair, we are at the very beginning of the semester and are still covering very basic material. The videos are good, I learn stuff in them. I think the breakout rooms could be good once we get them working as it is hard to interact as a large class. The google docs format doesn't seem to be the most effective, it just feels very much like a free for all. It would be better if there were pre-made slots on the doc for each group instead of just a large blank. But it is a good way for everyone to be able to contribute and put ideas down. Definitely creative and a great way of adapting to the online teaching format. 

Overall, I think the setup is very good for a fully online class as it clearly prioritizes student involvement. I really appreciate that. I think the synchronous component will improve as we get to new material and figure out breakout rooms. Thank you for doing a good job and asking for student input. It is really good. 

\end{document}


